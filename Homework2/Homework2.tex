\documentclass[12pt, a4paper, oneside]{ctexart}
\usepackage{amsmath, amsthm, amssymb, bm, graphicx, hyperref, mathrsfs, listings}

\title{\textbf{Homework2}}
\author{231220030 邢俊书}
\date{\today}
\linespread{1.5}
\setlength{\parindent}{0pt}
\NewDocumentEnvironment{solution}{o}{
    \IfValueTF{#1}
        {\def\solutionnumber{#1}} % 若提供编号,则使用自定义编号
        {\def\solutionnumber{??}} % 未提供编号则默认 ??
    \par\vspace{1em}\noindent\textbf{Solution \solutionnumber. }\newline
}{\par}

\begin{document}
\maketitle
\begin{solution}[4.1]
不妨记二叉树中总节点个数为 n,叶节点个数为 $n_0$,
有一个子节点的节点个数为 $n_1$,有两个子节点的节点个数为 $n_2$,
不难得出有如下等式成立:
\[
\begin{cases}
    n = n_0 + n_1 + n_2 \\
    n - 1 = n_1 + 2n_2
\end{cases}
\]
两式相减,得:$n_2 = n_0 + 1$。因为二叉树高度为 h,
所以有 $n \leq 2^{h + 1} - 1$ 成立,即 $n_0 + n_1 + n_2 \leq 2^{h + 1} - 1$,
又$n_0 + n_1 + n_2 = 2n_0 + n_1 - 1 \geq 2n_0 - 1$,因此 $2n_0 - 1 \leq 2^{h + 1} - 1$,
$n_0 \leq 2^h$,亦即 $L \leq 2^h$。
\end{solution}

\begin{solution}[4.4]
为了方便起见,不妨先约定一个 swap() 函数:
\begin{verbatim}
void swap(int& a, int& b) {
    int tmp = b;
    b = a;
    a = tmp;
}
\end{verbatim}
\newpage
(1) 以下给出的算法本质上是在简单情况下的归并排序:
\begin{verbatim}
void MySort(int a[]) {
    if (a[0] > a[1]) swap(a[0], a[1]);
    if (a[2] > a[3]) swap(a[2], a[3]);
    int b[4];
    int i = 0, j = 2, k = 0;
    while (i <= 1 && j <= 3) {
        if (a[i] <= a[j]) b[k++] = a[i++];
        else b[k++] = a[j++];
    }
    while (i <= 1) b[k++] = a[i++];
    while (j <= 3) b[k++] = a[j++];
    for (int i = 0; i <= 3; i++) {
        a[i] = b[i];
    }
    return;
}
\end{verbatim}
在 while 循环开始前进行了 2 次比较,在 while 循环开始后,由鸽笼原理可知,至多进行 3 次比较,
故该算法在最坏情况下可以只利用 5 次比较对 4 个元素进行排序。
\\(2) 不妨记待排序的五个数分别为 A、B、C、D、E。
先比较 A 与 B、C 与 D 的大小关系,不失一般性,假设 A > B、C > D,
再比较 A 与 C 的大小关系,不失一般性,假设 A > C,
至此进行了 3 次比较操作,得到了这样一个序列 A > C > D。
现在我们考虑将 E 插入到这个序列之中:易知,仅需先将 E 与 C 比较,再与 A 或 D 比较即可。
最后将 B 插入得到的 C、D、E 序列中,操作与将 E 插入 A > C > D 的序列之中类似。
故最坏情况下,该算法仅需 7 次比较,即可将 5 个元素排序。具体代码可见附件。
\\下证在最坏情况下,该算法具有最优性:
初始情况下,5 个元素的不同排列共有 $5! = 120$ 种,故决策树的叶节点至少应有 120 个,
若仅进行 6 次比较,至多有 $2^6 = 64$ 个叶节点,故最坏情况下,至少需要 7 次比较才能实现对 5 个元素的排序。
\end{solution}

\begin{solution}[4.8]
首先调用 nth-element() 函数以 O(n) 的时间复杂度计算数组的中位数,
随后部分划分数组,使得数组左半元素均小于中位数,右半元素均大于中位数,
再递归处理左右两部分直至将数组分为 k 段。
\\算法的时间复杂度满足 $f(n) = 2f(n / 2) + O(n)$,
易知,算法运行 logk 次后终止,故总时间复杂度为 O(nlogk)。
\end{solution}

\begin{solution}[4.9]
首先随机选取一个螺钉,将所有螺母与之匹配,可以找到与该螺钉匹配的螺母,
同时也将所有螺母分为两部分,一部分小于该螺钉,一部分大于该螺钉,
再取与该螺钉匹配的螺母,将剩余所有螺钉与之匹配,同样可以将螺钉分为两部分,
递归处理左右两部分的子问题即可解决问题。
\\算法每次都会选取一个 pivot 元素,再以 O(n) 的时间复杂度将所需处理的两组数据各自分为大小两部分,
故算法本质上等效于两次快排操作,时间复杂度为 $O(nlogn)$。
\end{solution}

\begin{solution}[4.11]
(1) 不妨假设存在 i、j,使得 (i, j) 为逆序对且 j - i > 2,则 A[i] 与 A[j] 之间至少存在两个元素,
记其中两个元素为 A[k]、A[l] 且 k < l,由逆序对性质可知,A[i] > A[j]。
\\若 A[k] > A[i],则 (k, j) 也是逆序对,
A[k] < A[l] < A[j] 必然不成立,(k, l) 与 (l, j) 之间必然存在一个逆序对,
与数组 A 至多有 2 个逆序对矛盾。
\\若 A[k] < A[i],则 (i, k) 也是逆序对,
若 A[k] < A[l] < A[j] 成立,则 (i, l) 也是逆序对,
否则 (k, l) 与 (l, j) 之间必然存在一个逆序对,
与数组 A 至多有 2 个逆序对矛盾。
\\综上,假设不成立,若 (i, j) 为逆序对,则 $j - i \leq 2$。
\\(2) 假设 (i, j) 为逆序对,若 j - i = 1,则 A[i] 与 A[j] 相邻;
若 j - i = 2,记中间元素为 A[k],则 A[i] < A[k] < A[j] 必然不成立,(i, k) 与 (k, j) 之间必然存在一个逆序对,
即无论 j - i 的取值,数组中必然存在一对逆序对,其两个元素相邻。故算法思路如下:
\\先遍历数组,找到第一个逆序对 (i, i + 1),此时 (i - 1, i + 1) 与 (i, i + 2) 之间可能存在逆序对。
首先交换 A[i] 与 A[i + 1],使得数组部分有序,交换后 (i - 1, i) 与 (i + 1, i + 2) 之间可能存在逆序对,
(i + 1, i + 2) 在向后遍历的过程中会被检查,故向前检查 A[i - 1] 与 A[i],
若 A[i - 1] > A[i],则交换 A[i - 1] 与 A[i],并退出循环;
否则继续向后进行比较,直到遍历完数组或找到并交换完第二个逆序对。
\\由于遍历完数组仅需 n - 1 次比较,且在整个遍历过程中,至多向前比较 1 次,故算法在最坏情况下的比较次数不超过 n 次。
\end{solution}

\begin{solution}[4.14]
先对所有单词进行预处理,将单词中所有字母以字母序重组,例如,“eat” 与 “tea” 都会被重组为 "aet",
再以每个单词重组后得到新字母序列为键值,将所有单词映射至哈希表中,最后遍历所有哈希表,找出其中元素大于等于 2 的即可。
\end{solution}

\begin{solution}[7.1]
暴力算法仅需要以两层循环遍历每一个二元组即可,易知时间复杂度为 $O(n^2)$。
\\若要优化时间复杂度,可以参照归并排序求解传统逆序对的思路,修改合并子问题时的判断条件即可,
可将时间复杂度优化至 O(nlogn),具体代码如下:
\newpage
\begin{verbatim}
int merge_sort(int A[], int l, int r) {
    int cnt = 0;
    if (l >= r) return 0;
    int mid = (l + r) >> 1;
    cnt += merge_sort(A, l, mid);
    cnt += merge_sort(A, mid + 1, r);
    int i = l, j = mid + 1, k = 0;
    int tmp[r - l + 1];
    while (i <= mid && j <= r) {
        if (A[i] <= C * A[j]) {
            tmp[k++] = A[i++];
        } // 修改的判断条件
        else {
            tmp[k++] = A[j++];
            cnt += mid - i + 1;
        }
    }
    while (i <= mid) tmp[k++] = A[i++];
    while (j <= r) tmp[k++] = A[j++];
    for (int s = l, t = 0; s <= r; s++, t++) {
        A[s] = tmp[t];
    }
    return cnt;
}
\end{verbatim}
\end{solution}

\newpage
\begin{solution}[7.4]
(1) 将一个长度为 mn 的有序数组与一个长度为 n 的有序数组合并的最坏时间代价为 c(m + 1)n,
故题述方案的时间复杂度为 $\sum_{i = 1}^{k - 1} c(i + 1)n = \frac{(k + 2)(k - 1)}{2}n = O(nk^2)$。
\\(2) 将需要合并的 k 个数组分为两堆,若堆中恰有两个数组,则以归并排序类似的方式合并两数组,
否则继续将堆进行划分,最后将已合并的两个堆合并。
时间复杂度满足,f(k) = 2f(k / 2) + nk,易知时间复杂度为 O(nklogk)。
\end{solution}

\begin{solution}[7.5]
(1) 不妨先约定树节点的结构体为:
\begin{verbatim}
struct TreeNode {
    int val;
    TreeNode* left;
    TreeNode* right;
};
\end{verbatim}
算法如下:
\begin{verbatim}
int getDepth(TreeNode* root) {
    int depth = 0;
    if (root == nullptr) return 0;
    if (root->left != nullptr) {
        depth = max(depth, getDepth(root->left) + 1);
    }
    if (root->right != nullptr) {
        depth = max(depth, getDepth(root->right) + 1);
    }
    return depth;
}
\end{verbatim}
(2) 算法如下:
\begin{verbatim}
int maxDiameter = 0;

int dfs(TreeNode* root) {
    if (!root) return 0;
    int leftDepth = dfs(root->left);
    int rightDepth = dfs(root->right);
    maxDiameter = max(maxDiameter, leftDepth + rightDepth);
    return max(leftDepth, rightDepth) + 1;
}

int treeDiameter(TreeNode* root) {
    maxDiameter = 0;
    dfs(root);
    return maxDiameter;
}
\end{verbatim}
\end{solution}

\begin{solution}[7.8]
(1) 先调用排序算法对所有点按照横坐标大小进行从小到大的排序,
接着从大到小开始遍历所有点,横坐标最大的必然是 maxima。向前遍历时,维护一个 y 来记录已遍历点中纵坐标最大的点,
若遍历过程中有点的纵坐标大于 y,则该点也是 maxima,同时更新 y;否则某点必然不是 maxima。
\\ 先以 O(n) 的时间代价计算横坐标的中位数,根据中位数将所有点划分为两部分,
递归解决左右两部分的 maxima。对于右半部分,其 maxima 必为局部的 maxima,对于左半部分,需要与右半部分维护的 y 进行比较。
\\(2) 算法不正确,因为递归的过程中划分不一定均匀,不可能每次递归都恰好将数组划分为四等块,故递归式并不满足。
\end{solution}

\begin{solution}[7.12]
(1) 分别遍历 k 行,统计每一行中 1 出现的次数。若为偶数,则缺失的比特串相应位置为 0;否则为 1。
\\(2) 不难发现,每遍历完一行,均可以将问题规模缩减至 $\frac{n - 1}{2}$。
被排除的 $\frac{n + 1}{2}$ 个元素往后的每一行,0 和 1 出现的次数均相等,否则由鸽笼原理可知,其中必然会存在相同元素。
时间复杂度满足 f(n) = f(n / 2) + n,由主定理可知,时间复杂度为 O(n)。
\end{solution}

\begin{solution}[14.1]
假设一个堆中共有 h 个元素,则堆的高度为 $\lceil log(h + 1) \rceil - 1$,
删除所有的叶子节点后的高度为 $\lceil log(\lfloor \frac{1}{2} h \rfloor + 1) \rceil - 1$,
余下的堆的高度必然比原来的堆少 1,故等式成立。
\end{solution}

\begin{solution}[14.2]
为方便讨论,不妨假定给定堆已被组织为二叉树形式,并约定树节点的结构体为:
\begin{verbatim}
struct TreeNode {
    int val;
    TreeNode* left;
    TreeNode* right;
};
\end{verbatim}
考虑维护一个最大堆 maxHeap,初始情况下,将给定堆的堆顶元素入堆,显然这个元素就是最大元素,
由堆的偏序关系可知,第 2 大元素仅可能为这个元素的子节点,将其子节点入堆并将 maxHeap 堆顶元素出堆。
如此不断地将 maxHeap 堆顶元素出堆并将其子节点入堆,即可保证所维护的堆中永远是所有可能的第 i 大的元素所构成的集合。
算法的时间复杂度为 $\sum_{i = 1}^{k} logi = logk! = O(klogk)$。
\\具体代码如下:
\begin{verbatim}
int get_kth_element(TreeNode* root, int k) {  
    if (!root) return -1;
    maxHeap.push(root);
    int cnt = 0;
    TreeNode* tmp = nullptr;
    while (cnt < k && !maxHeap.empty()) {
        tmp = maxHeap.top();
        maxHeap.pop();
        cnt++;   
        if (tmp->left) maxHeap.push(tmp->left);
        if (tmp->right) maxHeap.push(tmp->right);
    }    
    return tmp ? tmp->val : -1;
}
\end{verbatim}
\end{solution}

\begin{solution}[14.3]
先证 D-ARY-CHILD(i, j) 的正确性:
\\记 P(i):$\forall 1 \leq j \leq d$,D-ARY-CHILD(i, j) = d(i - 1) + j + 1。
\\显然 D-ARY-CHILD(1, j) = j + 1,所以 P(1) 成立。
\\假设当 $n\leq i$ 时,P(n) 成立,下证 P(i + 1) 成立。
\\不难发现 D-ARY-CHILD(i + 1, j) = D-ARY-CHILD(i, d) + j = d(i - 1) + d + 1 + j = di + j + 1,成立。
\\由数学归纳法可知,D-ARY-CHILD(i, j) 正确。
\\再证 D-ARY-PARENT(i) 的正确性:
\\显然根节点成立,接着考虑非根节点,由上述证明可知,对于任意下标为 n 的节点,都可以写为 n = d(i - 1) + j + 1,
代入 D-ARY-PARENT() 函数,$\lfloor \frac{i - 2}{d} + 1 \rfloor= \lfloor \frac{d(i - 1) + j - 1}{d} + 1 \rfloor = i$,正确性得证。
\end{solution}

\begin{solution}[14.4]
不妨先考虑完美二叉树的所有节点高度之和,假设树高度为 h,则所有节点的高度之和为 $\sum_{i = 1}^{h - 1} i * 2^{h - 1 - i} = 2^h - h - 1 = n - \lceil logn \rceil$。
此时若再加入一个节点,则高度之和恰变为了 n,即题目所需的取等的情况。
\\记 P(h):对于高度为 h 的堆,所有节点之和最多为 n - 1。
\\显然高度为 0 时,所有节点高度之和为 0,P(0) 成立。
\\假设当 $h \leq k$ 时,P(h) 成立,下证 P(k + 1) 成立。
\\显然一个堆的根节点的左右子堆大小分别为 $n_1$、$n_2$,
记左右子堆所有节点的高度之和分别为 $H_1$、$H_2$,则该堆所有节点的高度之和为 $H_1 + H_2 + \lceil logn_1 \rceil$。
对左右子树是不是完美二叉树进行讨论,发现会出现两种情况:
\\左子树为完美二叉树,则 $H_1 + H_2 + \lceil logn_1 \rceil \leq n_1 - \lceil logn_1 \rceil + n_2 - 1 + \lceil logn_1 \rceil < n_1 + n_2 + 1 - 1$,成立。
\\右子树为完美二叉树,则 $H_1 + H_2 + \lceil logn_1 \rceil \leq n_1 - 1 + n_2 - \lceil logn_1 \rceil + 1 + \lceil logn_1 \rceil = n_1 + n_2 + 1 - 1$,成立。
\\综上,在一个有 n 个节点的堆中,所有节点的高度之和最多为 n - 1。
\end{solution}

\begin{solution}[14.5]
首先以 k 个已排序链表的头节点创建一个最小堆 minHeap,每次将堆顶元素出堆并将相应节点的后继节点(如果存在)入堆,
这样可以保证 minHeap 中元素永远是剩余所有元素中最小元素的可能元素构成的集合。
最坏情况下,除剩余元素数小于 k 时外,每次维护堆的代价为 logk,执行 n 次后终止,故时间复杂度为 O(nlogk)。
\end{solution}

\begin{solution}[14.6]
维护两个堆 maxHeap、minHeap 分别存储较小的与较大的一半元素,
同时保证 $minHeap.size() \leq maxHeap.size() \leq minHeap.size() + 1$。
当输入元素数为偶数时,中位数为两个堆堆顶元素的平均数;当输入元素数为奇数时,中位数即为 maxHeap 的堆顶元素。
\begin{verbatim}
void addNum(int num) {
    if (maxHeap.empty() || num <= maxHeap.top()) {
        maxHeap.push(num);
    } 
    else minHeap.push(num);
    if (maxHeap.size() > minHeap.size() + 1) {
        minHeap.push(maxHeap.top());
        maxHeap.pop();
    } 
    else if (minHeap.size() > maxHeap.size()) {
        maxHeap.push(minHeap.top());
        minHeap.pop();
    }
}
    
double findMedian() {
    if (maxHeap.size() > minHeap.size()) {
        return maxHeap.top();
    }
    return (maxHeap.top() + minHeap.top()) / 2.0;
}
\end{verbatim}
删除操作可以考虑通过哈希表+延迟删除实现。
\end{solution}
\end{document}
