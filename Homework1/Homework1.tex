\documentclass[12pt, a4paper, oneside]{ctexart}
\usepackage{amsmath, amsthm, amssymb, bm, graphicx, hyperref, mathrsfs, listings}

\title{\textbf{Homework1}}
\author{231220030 邢俊书}
\date{\today}
\linespread{1.5}
\setlength{\parindent}{0pt}
\NewDocumentEnvironment{solution}{o}{
    \IfValueTF{#1}
        {\def\solutionnumber{#1}} % 若提供编号,则使用自定义编号
        {\def\solutionnumber{??}} % 未提供编号则默认 ??
    \par\vspace{1em}\noindent\textbf{Solution \solutionnumber. }\newline
}{\par}

\begin{document}
\maketitle
\begin{solution}[1.2]
(1) 代码如下:
\begin{verbatim}
int getMedian(int a, int b, int c) {
    if (a > b) {
        if (b > c) return b;
        else {
            if (a > c) return c;
            else return a;
        }
    }
    else {
        if (b > c) {
            if (a < c) return c;
            else return a;
        }
        else return c;
    }
}
\end{verbatim}
(2) 最坏情况下,需要比较 3 次;
\\平均情况下,需要比较 $\frac{1}{3} * 2 + \frac{2}{3} * 3 = \frac{8}{3}$ 次。
\\(3) 在最坏情况下,要找出三个不同整数的中位数至少需要进行 3 次比较。
\\显然,三个不同整数的大小关系共有 3! = 6 种不同情况,仅进行 2 次比较,只能得到 4 种情况;而进行 3 次比较,则可以得到 8 种情况,故至少需要进行 3 次比较。
\end{solution}

\begin{solution}[1.3]
(1) 取$U = \{1, 2, 3, 4, 5\}, S_1 = \{1, 2, 3\}, S_2 = \{2, 3, 4\}, S_3 = \{4, 5\}$,
题述算法得到的最小覆盖为 $\{S_1, S_2, S_3\}$,实际上最小覆盖为 $\{S_1, S_2\}$。
\\(2) 每次都从子集中选取覆盖最多未被覆盖元素的集合,直至所有元素均被覆盖,若遍历完所有子集均未覆盖所有元素,则不存在覆盖。
\\算法至多遍历完所有子集,所以总能终止;
\\不存在覆盖的情况下,算法正确性显然,故以下讨论均基于至少存在一个可行解的情况。
\\当 $|U| = 1$,则所有子集中至少有一个子集包含这个唯一元素,算法会选择该子集,正确性显然;
\\假设当 $|U| \leq n$ 时,算法总能找到一个可行解,下证当 $|U| = n + 1$ 时,算法可以找到一个可行解:
\\根据算法描述,算法首先会选取一个覆盖了最多未被覆盖元素的子集,假设该子集大小为 m,
则问题被转化为规模为 $n + 1 - m \leq n$ 的子问题,根据归纳假设,算法总能给出该子问题的一个可行解,所以当 $|U| = n + 1$ 时,算法可以找到一个可行解。
\\根据强数学归纳法,当输入存在可行解时,算法总能找到一个可行解。
\\综上,算法正确性得证。
\\(3) 不能保证总是得出最小覆盖。
\\例如取 $U = \{1, 2, 3, 4, 5, 6\}, S_1 = \{1, 2, 3, 4\}, S_2 = \{2, 3, 5\}, S_3 = \{1, 4, 6\}$,
\\根据 (2) 中算法得到的覆盖为 $\{S_1, S_2, S_3\}$,实际上最小覆盖为 $\{S_2, S_3\}$。
\end{solution}

\begin{solution}[1.7]
不难发现 $P(x) = ( \dots ((a_n x + a_{n - 1})x + a_{n - 2})x + \dots + a_1)x + a_0$。
\\记已循环次数为 k,当 k = 1 时,$p = a_n x + a_{n - 1}$;
\\假设当 $k = m$ 时,算法可以正确计算由内至外第 m 个括号内的值,
下证当 k = m + 1 时,算法可以正确计算出由内至外第 m + 1 个括号内的值:
\\由归纳假设,第 m + 1 次循环前,$p = a_n x^{m} + a_{n - 1} x^{m - 1} + \dots + a_{n - m}$,
则第 m + 1 次循环后,$p = a_n x^{m + 1} + a_{n - 1} x^m + \dots + a_{n - m - 1}$。
\\根据数学归纳法,算法可以正确计算出多项式的值。
\end{solution}

\begin{solution}[1.8]
(1) c = 2 是 (2) 的特殊情况,由下可知正确性显然。
\\(2) 不妨考虑对 z 进行归纳。
\\当 z = 0 时,根据算法有 INT-MULT(y, 0) = 0,算法可以正确计算出两数乘积;
\\当 $0 < z < c$ 时,INT-MULT(y, z) = INT-MULT(cy, 0) + yz = yz,算法可以正确计算出两数乘积;
\\假设当 $c \leq z \leq n$ 时,算法可以正确计算出两数乘积,下证当 z = n + 1 时,算法可以正确计算出两数乘积:
\\记 $n + 1 = sc + t, 0 \leq t < c$,由归纳假设可知,当 $z \leq n$ 时,算法总能正确计算出两数乘积,则有
\begin{align*}
    \text{INT-MULT}(y, n + 1) &= \text{INT-MULT}(cy, \lfloor \frac{n + 1}{c} \rfloor) + y * ((n + 1) \text{ mod } c) \\
    &= \text{INT-MULT}(cy, s) + yt \\
    &= y(sc + t) \\
    &= y(n + 1)  
\end{align*}
根据强数学归纳法,算法正确总能计算出两数乘积,算法正确。
\end{solution}

\newpage
\begin{solution}[1.9]
平均时间复杂度为
\begin{align*}
    A(n) &= \sum_{I \in D_n} Pr(I) * f(I) \\
    &= \frac{1}{4} * 10 + \frac{1}{2} * 20 + \frac{1}{8} * 30 + \frac{1}{8} * n \\
    &= \frac{n}{8} + 16.25
\end{align*}
\end{solution}

\begin{solution}[1.10]
(1) UNIQUE 算法用于判断数组中是否有两元素相等,最坏情况即不存在相等元素,需要遍历完所有 i、j,时间复杂度为 $O(n^2)$。
\\(2) 平均时间复杂度为 $A(n) = \frac{1 + 2 + \dots + \frac{n(n - 1)}{2}}{\frac{n(n - 1)}{2}} = \frac{n^2 - n + 2}{4}$。
\\(3) 数组中任意两个元素相等的概率为 $Pr(A[i] == A[j]) = k * (\frac{1}{k})^2 = \frac{1}{k}$,
不妨假设每次比较均消耗 1 个代价,则消耗了 i 个代价后算法终止的概率为 $(1 - \frac{1}{k})^{i - 1} * \frac{1}{k}$,
不难发现算法停止时消耗的代价 $X \sim g(\frac{1}{k})$,$EX = k$,
若 $k < \frac{n(n - 1)}{2}$,则时间复杂度近似于 $O(k)$;否则,时间复杂度近似于 $O(n^2)$。
\end{solution}

\begin{solution}[2.2]
对于任意 $2^k \leq n \leq 2^{k + 1} - 1, k \geq 0$,$\lceil log(n + 1) \rceil = k + 1 = \lfloor logn \rfloor + 1$。
\end{solution}

\begin{solution}[2.5]
(1) 2-tree 是 (2) 的特殊情况,由下可知等式成立。
\\(2) 假设总结点数为 n,则有
\[
\begin{cases}
    n = n_0 + n_1 + n_2 \\
    n - 1 = n_1 + 2n_2
\end{cases}
\]
消去 $n_1$ 即得:$n_0 = n_2 + 1$。
\end{solution}

\newpage
\begin{solution}[2.7]
为了方便讨论,不妨先约定三个渐进增长率 f(n)、g(n)、h(n)。
\\(1) O:f(n) = O(g(n))、g(n) = O(h(n)),
根据定义,$\lim_{n \rightarrow \infty} \frac{f(n)}{g(n)} = c_1 < \infty$ 且 $\lim_{n \rightarrow \infty} \frac{g(n)}{h(n)} = c_2 < \infty$,
则有 $\lim_{n \rightarrow \infty} \frac{f(n)}{h(n)} = c_1c_2 < \infty$,f(n) = O(h(n)),O 满足传递性。
\\$\Omega$:$f(n) = \Omega(g(n))$、$g(n) = \Omega(h(n))$,
根据定义,$\lim_{n \rightarrow \infty} \frac{f(n)}{g(n)} = c_1 > 0$ 且 $\lim_{n \rightarrow \infty} \frac{g(n)}{h(n)} = c_2 > 0$,
则有 $\lim_{n \rightarrow \infty} \frac{f(n)}{h(n)} = c_1 c_2 > 0$,$f(n) = \Omega(h(n))$,$\Omega$ 满足传递性。
\\$\Theta$:$f(n) = \Theta(g(n))$、$g(n) = \Theta(h(n))$,
根据定义,$\lim_{n \rightarrow \infty} \frac{f(n)}{g(n)} = c_1, 0 < c_1 < \infty$ 且 $\lim_{n \rightarrow \infty} \frac{g(n)}{h(n)} = c_2, 0 < c_2 < \infty$,
则有 $\lim_{n \rightarrow \infty} \frac{f(n)}{h(n)} = c_1 c_2, 0 < c_1 c_2 < \infty$,$f(n) = \Theta(h(n))$,$\Theta$ 满足传递性。
\\o:f(n) = o(g(n))、g(n) = o(h(n)),
根据定义,$\lim_{n \rightarrow \infty} \frac{f(n)}{g(n)} = 0$ 且 $\lim_{n \rightarrow \infty} \frac{g(n)}{h(n)} = 0$,
则有 $\lim_{n \rightarrow \infty} \frac{f(n)}{h(n)} = 0$,f(n) = o(h(n)),o 满足传递性。
\\$\omega$:$f(n) = \omega(g(n))$、$g(n) = \omega(h(n))$,
根据定义,$\lim_{n \rightarrow \infty} \frac{f(n)}{g(n)} = \infty$ 且 $\lim_{n \rightarrow \infty} \frac{g(n)}{h(n)} = \infty$,
则有 $\lim_{n \rightarrow \infty} \frac{f(n)}{h(n)} = \infty$,$f(n) = \omega(h(n))$,$\omega$ 满足传递性。
\\(2) O:$\lim_{n \rightarrow \infty} \frac{f(n)}{f(n)} = 1 < \infty$,f(n) = O(f(n)),O 满足自反性。
\\$\Omega$:$\lim_{n \rightarrow \infty} \frac{f(n)}{f(n)} = 1 > 0$,$f(n) = \Omega(f(n))$,$\Omega$ 满足自反性。
\\$\Theta$:$\lim_{n \rightarrow \infty} \frac{f(n)}{f(n)} = 1, 0 < 1 < \infty$,$f(n) = \Theta(f(n))$,$\Theta$ 满足自反性。
\\(3) 由 (1)、(2) 可知:$\Theta$ 满足自反性与传递性,故仅需再证其满足对称性即可。
$f(n) = \Theta(g(n))$,根据定义,$\lim_{n \rightarrow \infty} \frac{f(n)}{g(n)} = c, 0 < c < \infty$,
则有 $\lim_{n \rightarrow \infty} \frac{g(n)}{f(n)} = \frac{1}{c}, 0 < \frac{1}{c} < \infty$,$g(n) = \Theta(f(n))$,$\Theta$ 满足对称性,
综上,$\Theta$ 是一个等价关系。
\\(4) $\Rightarrow$:因为 $f(n) = \Theta(g(n))$,故 $\lim_{n \rightarrow \infty} \frac{f(n)}{g(n)} = c, 0 < c < \infty$,
则有 $\lim_{n \rightarrow \infty} \frac{f(n)}{g(n)} = c > 0 \land \lim_{n \rightarrow \infty} \frac{f(n)}{g(n)} = c < \infty$,
即 $f(n) = O(g(n)) \land f(n) = \Omega(g(n))$;
\newpage
$\Leftarrow$:因为 $f(n) = O(g(n)) \land f(n) = \Omega(g(n))$,故 $\lim_{n \rightarrow \infty} \frac{f(n)}{g(n)} = c_1 > 0 \land \lim_{n \rightarrow \infty} \frac{f(n)}{g(n)} = c_2 < \infty$,
显然 $c_1 = c_2$,则有 $\lim_{n \rightarrow \infty} \frac{f(n)}{g(n)} = c, 0 < c < \infty$,
即 $f(n) = \Theta(g(n))$。
\\综上,得证。
\\(5) $\Rightarrow$:因为 f = O(g),故 $\lim_{n \rightarrow \infty} \frac{f(n)}{g(n)} = c < \infty$,
则有 $\lim_{n \rightarrow \infty} \frac{g(n)}{f(n)} = \frac{1}{c} > 0$,$g = \Omega(f)$;
\\$\Leftarrow$:因为 $g = \Omega(f)$,故  $\lim_{n \rightarrow \infty} \frac{g(n)}{f(n)} = c > 0$,
则有 $\lim_{n \rightarrow \infty} \frac{f(n)}{g(n)} = \frac{1}{c} < \infty$,$f = O(g)$。
\\综上,$f = O(g) \text{ iff } g = \Omega(f)$。
\\$\Rightarrow$:因为 f = o(g),故 $\lim_{n \rightarrow \infty} \frac{f(n)}{g(n)} = 0$,
则有 $\lim_{n \rightarrow \infty} \frac{g(n)}{f(n)} = \infty$,$g = \omega(f)$;
\\$\Leftarrow$:因为 $g = \omega(f)$,故  $\lim_{n \rightarrow \infty} \frac{g(n)}{f(n)} = \infty$,
则有 $\lim_{n \rightarrow \infty} \frac{f(n)}{g(n)} = 0$,$f = o(g)$。
\\综上,$f = o(g) \text{ iff } g = \omega(f)$。
\\(6) $\forall f(n) \in o(g(n))$,$lim_{n \rightarrow \infty} \frac{f(n)}{g(n)} = 0 \neq \infty$,故 $f(n) \notin \omega(g(n))$,
\\$o(g(n)) \cap \omega(g(n)) = \emptyset$;
\\$\forall f(n) \in \Theta(g(n))$,$lim_{n \rightarrow \infty} \frac{f(n)}{g(n)} = c > 0$,故 $f(n) \notin o(g(n))$,
\\$\Theta(g(n)) \cap o(g(n)) = \emptyset$;
\\$\forall f(n) \in \Theta(g(n))$,$lim_{n \rightarrow \infty} \frac{f(n)}{g(n)} = c < \infty$,故 $f(n) \notin \omega(g(n))$,
\\$\Theta(g(n)) \cap \omega(g(n)) = \emptyset$。
\end{solution}

\begin{solution}[2.8]
(1) 渐进增长率从低到高依次为:
\\$logn < n < nlogn < n^2 = n^2 + logn < n^3 < n - n^3 + 7n^5 < 2^n$。
\\(2) 渐进增长率从低到高依次为:
\\$loglogn < lnn = logn < (logn)^2 < \sqrt{n} < n < nlogn < n^{1 + \epsilon} < n^2 = n^2 + logn < n^3 < n - n^3 + 7n^5 < 2^{n - 1} = 2^n < e^n < n!$。
\end{solution}

\begin{solution}[2.16]
(1) $f(n) = 1 = n^0 = O(n^{log_3 2 - 0.5})$,$T(n) = \Theta(n^{log_3 2})$。
\\(2) 为方便讨论,不妨假设 $n = 2^k$,$k = logn$,则有
\begin{align*}
    T(n) &= T(n / 2) + clogn \\
    &= T(n / 4) + clog n + clog \frac{n}{2} \\
    &= \dots \\
    &= T(1) + clog^2 n - c(log2 + log4 + \dots + log(2^{k})) \\
    &= \frac{c}{2} log^2 n - \frac{c}{2} logn + 1
\end{align*}
$T(n) = \Theta(log^2 n)$。
\\(3) $f(n) = cn = \Omega(n^{log_2 1 + 0.5})$,$f(n / 2) = \frac{cn}{2} < \frac{2}{3} f(n)$,$T(n) = \Theta(n)$。
\\(4) $f(n) = cn = \Theta(n^{log_2 2})$,$T(n) = \Theta(nlogn)$。
\\(5) 为方便讨论,不妨假设 $n = 2^k$,$k = logn$,则有
\begin{align*}
    T(n) &= 2T(n / 2) + cnlogn \\
    &= 4T(n / 4) + cnlogn + cnlog \frac{n}{2} \\
    &= \dots \\
    &= nT(1) + cnlog^2 n - cn(log2 + log4 + \dots + log2^k) \\
    &= n + \frac{c}{2} cnlog^2 n - \frac{c}{2}nlogn
\end{align*}
$T(n) = \Theta(nlog^2 n)$。
\\(6) 为方便讨论,不妨假设 $n = 3^k$,$k = log_3 n$,则有
\begin{align*}
    T(n) &= 3T(n / 3) + nlog^3 n \\
    &=  9T(n / 9) + nlog^3 n + nlog^3 \frac{n}{3} \\
    &= \dots \\
    &= nT(1) + \sum_{i = 0}^{k} nlog^3 \frac{n}{3^i}
\end{align*}
$T(n) = \Theta(nlog^4 n)$。
\newpage
(7) $f(n) = cn^2 = \Omega(n^{log_2 2 + 0.5})$,$2f(n / 2) = \frac{cn^2}{2} < \frac{2}{3} f(n)$,$T(n) = \Theta(n^2)$。
\\(8) $f(n) = n^{3 / 2}logn = \Omega(n^{log_5 7 + 0.2})$,$49T(n / 25) < \frac{49}{125} n^{3 / 2} logn = \frac{49}{125} f(n)$,$T(n) = \Theta(n^{3 / 2} logn)$。
\\(9) $T(n) = T(n - 1) + 2 = T(1) + 2n - 2 = 2n - 1$,$T(n) = \Theta(n)$。
\\(10) 不妨猜测 $\frac{1}{c + 1} n^{c + 1} \leq T(n) \leq n^{c + 1}$,代入可得:
\[
    T(n) = T(n - 1) + n^c \leq (n - 1)^{c + 1} + n^c \leq (n - 1) n^c + n^c = n^{c + 1}
\]
\begin{align*}
    T(n) &= T(n - 1) + n^c \\
    &\geq \frac{1}{c + 1} (n - 1)^{c + 1} + n^c \\
    &= \frac{1}{c + 1} (n^{c + 1} - \binom{c + 1}{1} n^c + \binom{c + 1}{2} n^{c - 1} + \dots + (-1)^{c + 1}) + n^c \\
    &> \frac{1}{c + 1} n^{c + 1}
\end{align*}
猜想成立,故 $T(n) = \Theta(n^{c + 1})$。
\\(11) $T(n) = T(n - 1) + c^n = \sum_{i = 2}^{n} c^n + 1, c > 1$,$T(n) = \Theta(c^n)$。
\\(12) 不妨猜测 $\frac{1}{4} n^4 \leq T(n) \leq n^4$,代入可得:
\begin{align*}
    T(n) &= T(n - 2) + 2n^3 - 3n^2 + 3n - 1 \\
    &\leq (n - 2)^4 + 2n^3 - 3n^2 + 3n - 1 \\
    &= n^4 - 6n^3 + 21n^2 - 29n + 16 \\
    &< n^4
\end{align*}
\begin{align*}
    T(n) &= T(n - 2) + 2n^3 - 3n^2 + 3n - 1 \\
    &\geq \frac{1}{4} (n - 2)^4 + 2n^3 - 3n^2 + 3n - 1 \\
    &= \frac{1}{4} n^4 + 3n^2 - 5n + 2 \\
    &> \frac{1}{4} n^4
\end{align*}
猜想成立,故 $T(n) = \Theta(n^4)$。
\\(13) 不妨猜测 $4n \leq T(n) \leq 8n$,代入可得:
\[
    T(n) = T(n / 2) + T(n / 4) + T(n / 8) + n \leq 7n + n \leq 8n
\]
\[
    T(n) = T(n / 2) + T(n / 4) + T(n / 8) + n \geq \frac{9}{2} n \leq 4n
\]
猜想成立,故 $T(n) = \Theta(n)$。
\end{solution}

\begin{solution}[2.18]
由题意得:
\begin{align*}
    T(n) &= n^{\frac{1}{2}} T(n^{\frac{1}{2}}) + cn \\
    &= n^{\frac{1}{2} + \frac{1}{4}} T(n^{\frac{1}{4}}) + 2cn \\
    &= n^{\frac{1}{2} + \frac{1}{4} + \frac{1}{8}} T(n^{\frac{1}{8}}) + 3cn \\
    &= \dots \\
    &= n^{\sum_{i = 1}^{k} \frac{1}{2^i}} T(n^{\frac{1}{2^k}}) + kcn \\
    &= \Theta(kcn)
\end{align*}
显然,$n^{\frac{1}{2^k}} = C$,得:$k = log(logn - logC)$,故 $T(n) = \Theta(nloglogn)$。
\end{solution}

\begin{solution}[2.19]
a = 2,b = 2,f(n) = nlogn。
\end{solution}

\begin{solution}[2.22]
(1) 两个算法的输出均为数组中最小元素。
\\(2) 两个算法的时间复杂度均为 $\Theta(n)$。
\end{solution}

\newpage
\begin{solution}[2.24]
MYSTERY:
\begin{align*}
    r &= \sum_{i = 1}^{n - 1} \sum_{j = i + 1}^{n} \sum_{k = 1}^{j} 1 \\
    &= \sum_{i = 1}^{n - 1} \sum_{j = i + 1}^{n} j \\
    &= \sum_{i = 1}^{n - 1} \frac{n^2 + n - i^2 - i}{2} \\
    &= \frac{(n - 1)n(n + 1)}{3}
\end{align*}
最坏情况下的运行时间为 $O(n^3)$。
\\PERSKY:
\begin{align*}
    r &= \sum_{i = 1}^{n} \sum_{j = 1}^{i} \sum_{k = j}^{i + j} 1 \\
    &= \sum_{i = 1}^{n} \sum_{j = 1}^{i} (i + 1) \\
    &= \sum_{i = 1}^{n} i(i + 1) \\
    &= \frac{n(n + 1)(n + 2)}{3}
\end{align*}
最坏情况下的运行时间为 $O(n^3)$。
\\PRESTIFEROUS:
\begin{align*}
    r &= \sum_{i = 1}^{n} \sum_{j = 1}^{i} \sum_{k = j}^{i + j} \sum_{l = 1}^{i + j - k} 1 \\
    &= \sum_{i = 1}^{n} \sum_{j = 1}^{i} \sum_{k = j}^{i + j} (i + j - k) \\
    &= \sum_{i = 1}^{n} \sum_{j = 1}^{i} \frac{i^2 + i}{2} \\
    &= \sum_{i = 1}^{n} \frac{i^3 + i^2}{2} \\
    &= \frac{n(n + 1)(3n^2 + 7n + 2)}{24}
\end{align*}
最坏情况下的运行时间为 $O(n^4)$。
\\CONUNDRUM:
当 n 为偶数时,
\begin{align*}
    r &= \sum_{i = 1}^{\frac{n}{2}} \sum_{j = i + 1}^{n + 1 - i} \sum_{k = i + j - 1}^{n} 1 \\
    &= \sum_{i = 1}^{\frac{n}{2}} \sum_{j = i + 1}^{n + 1 - i} (n - i - j + 2) \\
    &= \frac{1}{12} n^3 + \frac{1}{8} n^2 - \frac{1}{12} n
\end{align*}
当 n 为奇数时,
\begin{align*}
    r &= \sum_{i = 1}^{\frac{n - 1}{2}} \sum_{j = i + 1}^{n + 1 - i} \sum_{k = i + j - 1}^{n} 1 \\
    &= \sum_{i = 1}^{\frac{n - 1}{2}} \sum_{j = i + 1}^{n + 1 - i} (n - i - j + 2) \\
    &= \frac{1}{12} n^3 + \frac{1}{8} n^2 - \frac{1}{12} n - \frac{1}{8}
\end{align*}
最坏情况下的运行时间为 $O(n^3)$。
\end{solution}

\newpage
\begin{solution}[3.2]
(1) 首先对内层循环做证明,循环不变量: 第 j 次循环开始前, $A[1 \dots j]$ 部分的最大值是 A[j]。
\\Initialize: j = 1 的内层循环开始前,$A[1 \dots 1]$ 部分的最大值显然是 A[1]。
\\Maintenance: 假设 j = k 的循环开始前,数组 $A[1 \dots k]$ 部分的最大值为 A[k]。
在此次循环中,如果 A[j] > A[j + 1],则会交换这两个元素,循环结束后有 $A[j + 1] = max\{A[1 \dots j], A[j + 1]\} = max\{A[1 \dots j + 1]\}$,满足循环不变式。
\\Termination: 当 j = i - 1 次循环结束后,数组 $A[1 \dots i]$ 中的最大元素就是 A[i]。
\\接下来证明外层循环,循环不变量: 第 i 次循环开始前,数组 $A[i + 1 \dots n]$ 部分是数组 $A[1 \dots n]$ 升序排列后的对应部分。
\\Initialize: 在 i = n 的循环开始前,$A[n + 1 \dots n]$ 为空,满足。
\\Maintenance: 假设在 i = k 的循环开始前,不变式满足。
在此次循环中,由上述结论,内层循环会选出 $A[1 \dots k]$ 部分的最大值置于 A[k] 处,故循环后满足循环不变式。
\\Termination: 在 i = 2 的循环结束后,数组 $A[2 \dots n]$ 部分已经排好序,而显然 A[1] 就是第 1 小的元素,因此 $A[1 \dots n]$ 均已排好序。
\\(2) 无论内层循环还是外层循环,均不会提前终止,
总共需要比较 $\sum_{k = 1}^{n - 1} k = \frac{n(n - 1)}{2}$,
故最坏情况与平均情况时间复杂度均为 $\Theta(n^2)$。
\\(3) 最坏情况下,输入为完全倒序的数组,无论是否记录每次循环最后交换位置,都会进行 i - 1 次比较,故不会影响最坏情况;
平均情况下,大致可以减少一半的时间复杂度,但是整体时间复杂度仍为 $\Theta(n^2)$。
\end{solution}

\newpage
\begin{solution}[3.5]
修改后代码如下:
\begin{verbatim}
void PREVIOUS_LARGER(int a[], int n) {
    memset(P, 0, n + 1); // a 下标从 1 开始
    stack<int> s;
    for (int i = 1; i <= n; i++) {
        while (!s.empty() && a[s.top()] <= a[i]) s.pop();
        if (!s.empty()) P[i] = s.top();
        else P[i] = 0;
        s.push(i);
    }
    return;
}
\end{verbatim}
考虑如下循环不变式,每次循环开始前,栈中元素满足:1. 从底至顶依次递减;2. 栈顶元素为左侧最近的较大元素的索引。
\\Initialize:栈中没有元素,满足循环不变式;
\\Maintenance:假设第 i 轮循环开始之前满足循环不变式,
由算法可知,循环开始后会弹出栈中所有不大于 a[i] 的元素,最后将 a[i] 入栈,
由于循环开始前栈中元素自底至顶依次递减,故弹栈操作会在遇到第一个大于 a[i] 的元素或栈空后停止,
保证了栈中元素仍然自底至顶依次递减,且栈顶元素为左侧最近的较大元素的索引。
\\Termination:由循环不变式可知,栈中每个元素的下一个元素均是其左侧最近的较大元素的索引,
因而每次循环后均能正确地为 P[i] 赋值。
\\对于每个元素,至多入栈、出栈一次,故时间复杂度为 $\Theta(n)$。
\end{solution}

\newpage
\begin{solution}[3.6]
为了方便起见,不妨先约定 swap() 作为交换两个变量值的函数。
\\(1) 时间复杂度为 $O(n^2)$,空间复杂度为 $O(1)$:
\begin{verbatim}
void My_reverse(int a[], int n, int k) {
    for (int i = k + 1; i <= n; i++) {
        for (int j = i; j >= i - k + 1; j--) {
            swap(a[j - 1], a[j]);
        }
    }
    return;
}
\end{verbatim}
(2) 时间复杂度为 $O(n)$,空间复杂度为 $O(n)$:
\begin{verbatim}
void My_reverse(int a[], int n, int k) {
    int b[n + 1]; // a 下标从 1 开始
    memset(b, 0, n + 1);
    for (int i = 1; i <= k; i++) {
        b[n - k + i] = a[i]; 
    }
    for (int i = k + 1; i <= n; i++) {
        b[i - k] = a[i];
    }
    for (int i = 1; i <= n; i++) {
        a[i] = b[i];
    }
    return;
}
\end{verbatim}
\newpage
(3) 时间复杂度为 $O(n)$,空间复杂度为 $O(1)$:
\begin{verbatim}
void My_reverse(int a[], int n, int k) {
    int b[n + 1]; // a 下标从 1 开始
    memset(b, 0, n + 1);
    for (int i = 1; i <= k / 2; i++) {
        swap(a[i], a[k - i + 1]);
    }
    for (int i = k + 1; i <= (k + 1 + n) / 2; i++) {
        swap(a[i], a[n - i + k + 1]);
    }
    for (int i = 1; i <= n / 2; i++) {
        swap(a[i], a[n - i + 1]);
    }
    return;
}
\end{verbatim}
\end{solution}

\begin{solution}[3.8]
(1) 0 个或 1 个。
假设存在大于等于 2 个名人,任取其中两人 A、B,由名人的定义可知:A 不认识 B,与 B 是名人矛盾。
\\(2) 定义集合 S 为所有名人候选者名单,每次从中任取两人 A、B,询问 A 是否认识 B。
若不认识,则 B 不可能是名人;若认识,则 A 不可能是名人。
每次询问均可以从集合 S 中删除一人,经过 n - 1 次询问,即可得出名人,时间复杂度为 $\Theta(n)$。
\end{solution}

\newpage
\begin{solution}[3.9]
(1) $O(n^3)$:
\begin{verbatim}
int MaxSubarr(int S[], int n) {
    int max = INT_MIN;
    for (int i = 0; i < n; i++) {
        for (int j = i; j < n; j++) {
            int sum = 0;
            for (int k = i; k <= j; k++) sum += S[k];
            if (sum >= max) max = sum;
        }
    }
    return max;
}
\end{verbatim}
(2) $O(n^2)$:
\begin{verbatim}
int MaxSubarr(int S[], int n) {
    int max = INT_MIN;
    for (int i = 0; i < n; i++) {
        int sum = 0;
        for (int j = i; j < n; j++) {
            sum += S[j];
            if (sum >= max) max = sum;
        }
    }
    return max;
}
\end{verbatim}
\newpage
(3) 将整个数组分为左右两部分,递归求解每个部分的最大和连续子序列,
再以 O(n) 的时间代价求解横跨左右两部分的最大和连续子序列。
\\(4) $O(n)$:
\begin{verbatim}
int MaxSubarr(int S[], int n) {
    int sum = 0, max = INT_MIN;
    for (int i = 0; i < n; i++) {
        sum += S[i];
        if (sum < 0) sum = 0;
        if (sum >= max) max = sum;
    }
    return max;
}
\end{verbatim}
(5) 动态规划:
\begin{verbatim}
int MaxSubarr(int S[], int n) {
    int ans = 0, sum = 0;
    for (int i = 0; i < n; i++) {
        sum = max(sum + S[i], S[i]);
        ans = max(ans, sum);
    }
    return ans;
}    
\end{verbatim}
\end{solution}
\end{document}
